% Created 2015-02-28 Sat 15:32
\documentclass[11pt]{article}
\usepackage{graphicx}
\usepackage{float}
\usepackage{acl2015}
\usepackage{times}
\usepackage{url}
\usepackage{latexsym}
\usepackage{amsmath}
\usepackage{amssymb}
\usepackage{fontspec}
\usepackage{xunicode}

\usepackage{forest}
\DeclareMathOperator*{\argmin}{arg\,min}
\DeclareMathOperator*{\argmax}{arg\,max}
\newcommand{\BigO}[1]{\ensuremath{\operatorname{O}\bigl(#1\bigr)}}
\date{\today}
\title{CKY Parsing With Independence Constraints}
\begin{document}

\maketitle

\begin{abstract}
The CKY algorithm is an important component in many natural language
parsers. We propose a novel type of constraint for context-free
parsing called independence constraints. Based on the concept
of independence between words, we show how these constraints can be
used to reduce the work done in the CKY algorithm. We demonstrate a
classifier which can be used to identify boundaries between
independent words in a sentence using only surface features, and show
that it can be used to speed up a CKY parser. We investigate the
tradeoff between speed and accuracy, and indicate directions for
improvement.
\end{abstract}

\section{Introduction}
\label{sec-1}



The CKY algorithm is a \BigO{|G|n^3} dynamic programming
algorithm for finding all of the possible derivations of a sentence in
a context-free language. Its complexity depends on both the sentence
length $n$ and the size of the grammar $|G|$. Methods for improving
parse accuracy typically increase the size of the grammar 
\cite{Klein2003,Petrov2007} or even the exponent of $n$ \cite{Eisner1999}. 
More powerful “deep” grammar formalisms multiply the computational
complexity even more \cite{Bangalore1999}.

A common technique for speeding up such parsers is coarse-to-fine
parsing, where input is first parsed using a much simpler (and thus
smaller) grammar, and the content of the chart is then used to
constrain the search over the final grammar
\cite{Torisawa2000,Charniak2005,Petrov2007}. Even with a much smaller
grammar, the CKY algorithm may be expensive—Roark et al. (2012)
report that the initial CKY step in the Berkeley Parser takes half of
the total parse time.

Other techniques can be used to prune cells in the chart. Roark et al.
(2012) use a finite-state model to label words that do/don’t begin/end
spans, and skip cells that don’t satisfy the labels. Bodenstab et al.
(2011) directly apply a classifier to each cell to decide how many
spans to keep. Both approaches reduce the work done by the parser
while preserving accuracy.

We propose a novel type of top-down constraint for a CFG parser that
we call independence constraints, described in Section 2. In Section 3
we show how the CKY algorithm can be easily modified to accomodate
these constraints, and in Section 4 we describe a classifier which can
provide the constraints to a parser. We integrate the constraints into
the Stanford Parser CKY implementation and show the results in Section 5.

\section{Independence Constraints}
\label{sec-2}

\begin{figure}
\begin{forest}
  [S
   [NP [DT [ $_0$ This $_1$]]]
   [VP
    [VB [is $_2$]]
    [NP [DT [an $_3$]]
        [NN [example $_4$]]]]
   [{.} [{.} $_5$]]
  ]
\end{forest}
\caption{In this tree ‘This’ and ‘is’ are independent, while ‘is’ and ‘an’ are not.}
\label{fig:independence}
\end{figure}

We propose a concept we call \textbf{independence}. Given a sentence $s = w_1
w_2 \dots w_n$ and a context-free derivation (parse tree) $t$ of $s$,
words $w_i$ and $w_{i+1}$ are \textbf{independent} if every node in $t$ that
dominates both $w_i$ and $w_{i+1}$ also dominates $w_1$ and $w_n$.
Furthermore, if $w_i$ and $w_{i+1}$ are independent, then $\forall
j,k$ s.t. $j \leq i$ and $k > i$, $w_j$ and $w_k$ are independent.
Less formally, if the children of the top node of a parse tree were
split into separate subtrees, two words are independent if they would
end up in different subtrees.

An example is shown in Figure \ref{fig:independence}. Here, ‘This’ and
‘is’ are independent, as are ‘example’ and ‘.’. The independent spans
are (‘This’), (‘is’, ‘an’, ‘example’), and (‘.’), with boundaries 1
and 4. The independent spans and independent span boundaries can be
derived straightforwardly from the definition of independent words:
the locations between consecutive words which are independent are the
independent span boundaries, and the independent spans are simply the
spans in between consecutive boundaries.

\section{Modifying The CKY Algorithm With Independence Constraints}
\label{sec-3}

Conceptually, if a CKY parser knows the locations of the
independent span boundaries for a sentence, it can perform the normal
CKY algorithm for each independent span separately, and simply join
the spans at the top of the tree to finish the parse, thereby avoiding
work which would otherwise be done while still obtaining the desired
1-best parse. Two issues make the task more complicated than this.

The first complication is that if we assume that the independent boundaries
will be identified automatically, we must allow for errors. If a
location which is not an independent span boundary is given as one,
the parser will make an error it would not have otherwise. On the
other hand, if a location which is an independent span boundary is not
marked as such, the parser may account for this at the cost of not
achieving the minimum computation possible. By allowing for this
second type of error, the algorithm is made more robust, and allows
the independent boundary identification step to prioritize precision
over recall to lessen negative impact on the parser’s accuracy.

The second issue is caused by the binarization of the context-free
grammar used in the CKY algorithm. Because the CKY algorithm requires
a binary grammar, any rules in the original grammar that have more
than two symbols on the right-hand side must be converted into a
sequence of binary rules. The extra rules created in this process are
called intermediate or synthetic rules. The topmost span in particular
will usually need to be constructed in several steps; however, the
bottom-up nature of the algorithm means that it is impossible to
determine whether a given synthetic rule at any position in the CKY
chart will eventually be used to construct a complete span at the top
of the tree (for arbitrary grammars—if the grammar is left- or
right-binarized, only cells on the left or right edge of the chart can
affect the top span).

The combination of these two issues means that in order to correctly
parse a sentence when an independent span boundary is missing from the
input the modified CKY algorithm must process synthetic rules even at
positions in the chart that cross a boundary.

Thus in the modified algorithm, cells which do not cross an
independent boundary are processed normally, and in cells which do
cross a boundary the algorithm will avoid looping over complete
binary rules. While boundary-crossing cells depend on non-crossing
cells, the reverse is not the case; thus the non-crossing cells can
all be processed before the crossing cells, or the cells can be looped
over in the regular order, with a check inside the loop.

\subsection{How much work can we expect to save?}
\label{sec-3-1}

For our purposes, we can consider the amount of work done in the CKY algorithm
to be the number of binary edges visited in the inner loop. For each cell
the algorithm iterates over the binary rules in the grammar, calculating the
probability of the left-hand-side at each split point. The number of these
binary edges is

\begin{equation*}
|G|\left[\frac{n^3}{6} - \frac{n}{6}\right]
\end{equation*}

The amount of work saved depends on the number and locations of the
independent span boundaries, as well as the proportion of
non-synthetic edges in the grammar, denoted $\frac{|G_r|}{|G|}$. We
can consider two idealized scenarios: one boundary at $\frac{n}{K}$,
and $K-1$ boundaries at $\frac{n}{K}, \frac{2n}{K}, \dots,
\frac{(K-1)n}{K}$, where $K$ is an integer $1 < K < n$.

For the first case, the ratio of work saved approaches

\begin{equation*}
\frac{|G_r|}{|G|} \left[ \frac{3}{K} - \frac{3}{K^2} \right]
\end{equation*}

as $n$ grows. This limit converges quickly for $n \ge 10$. If we
approximate $|G_r|/|G|$ as 0.5 (for the grammar used by the parser in
Section \ref{sec:parser}, it is $\approx .54$), then for
$K=2,3,4,\dots$, the values are $\frac{3}{8}, \frac{1}{3},
\frac{3}{32}, \dots$. Intuitively, for one boundary, the best location
is exactly in the center of the sentence, and the upper limit on how
much work is saved is about 37\%.

For the case of $K-1$ boundaries equally spaced, the ratio is

\begin{equation*}
\frac{|G_r|}{|G|}\frac{K^2 - 1}{K^2}
\end{equation*}

The values for $K=2,3,4,\dots$ are $\frac{3}{8}, \frac{4}{9},
\frac{15}{32}, \dots$. Clearly, the smaller pieces a sentence can be
divided into the less work the parser will do; however, realistically
most sentences will not have a large number of independent spans, and
they will not be equal in length. We might take $K=3$ as best-case
estimate, giving us about 44\%. Thus we can guess that a parser will be
able to save around 35-45\% of the work it does in the CKY algorithm
loop (not counting unary rules) by using independence constraints.

\section{Classifying Independent Span Boundaries}
\label{sec-4}

In order to use independence constraints in a parser, we need to be
able to identify boundaries between independent words in a sentence
using only surface features (words and part-of-speech tags). We
created a binary classifier which, given a POS-tagged sentence and a
position between two words, decides whether those two words are
independent or not. Our classifier currently uses only POS tags as
features. We used \texttt{opal} \cite{Yoshinaga2010}, a tool for fast online
classification, to train and test the models, training on sentences
from Penn Treebank section 02-21 and testing on section 22. We set
opal to use the passive-aggressive perceptron update, and output
probabilities in order to use a threshold to trade off precision and
recall.

\subsection{Features}
\label{sec-4-1}

\begin{table*}[tbp]
%\resizebox{12cm}{!}{

\begin{tabular}{lrrrrrrrrrr}
Features & \#feats & Acc & Prec & Rec & F$_{\text{1}}$ & F$_{\text{0.5}}$ & TP & FP & FN & TN\\
\hline
p & 37001 & 93.71 & 80.73 & 70.49 & 75.27 & 78.45 & 3679 & 878 & 1540 & 32320\\
P$_{\text{0}}$ & 33167 & 87.16 & 51.69 & 83.98 & 63.99 & 55.99 & 4383 & 4097 & 836 & 29101\\
\hline
p,P$_{\text{0}}$ & 70168 & 95.21 & 87.38 & 75.65 & 81.09 & 84.75 & 3948 & 570 & 1271 & 32628\\
p,P$_{\text{1}}$ & 37055 & 94.81 & 78.38 & \textbf{85.38} & 81.73 & 79.69 & 4456 & 1229 & 763 & 31969\\
p,P$_{\text{2}}$ & 39336 & 95.34 & 84.25 & 80.76 & 82.47 & 83.53 & 4215 & 788 & 1004 & 32410\\
p,P$_{\text{3}}$ & 46861 & 95.04 & 89.47 & 71.95 & 79.76 & 85.31 & 3755 & 442 & 1464 & 32756\\
\hline
p,P$_{\text{0}}$,P$_{\text{1}}$ & 70222 & \textbf{95.48} & 88.95 & 76.16 & 82.06 & \textbf{86.06} & 3975 & 494 & 1244 & 32704\\
p,P$_{\text{0}}$,P$_{\text{2}}$ & 72503 & 95.09 & 88.28 & 73.60 & 80.27 & 84.89 & 3841 & 510 & 1378 & 32688\\
p,P$_{\text{0}}$,P$_{\text{3}}$ & 80028 & 94.84 & 88.81 & 70.99 & 78.91 & 84.56 & 3705 & 467 & 1514 & 32731\\
\hline
p,P$_{\text{1}}$,P$_{\text{2}}$ & 39390 & 95.27 & 80.99 & 85.21 & \textbf{83.04} & 81.80 & 4447 & 1044 & 772 & 32154\\
p,P$_{\text{1}}$,P$_{\text{3}}$ & 41553 & \textbf{95.44} & 89.05 & 75.74 & 81.86 & \textbf{86.03} & 3953 & 486 & 1266 & 32712\\
\hline
p,P$_{\text{0}}$,P$_{\text{1}}$,P$_{\text{2}}$,P$_{\text{3}}$ & 82417 & 95.35 & 86.89 & 77.49 & 81.92 & 84.83 & 4044 & 610 & 1175 & 32588\\
\end{tabular}

%}
\caption{Results of classifier using different combinations of features.}
\label{tbl:feature-evaluation}
\end{table*}

We use only part-of-speech tags to create features for the classifier
(adding lexical or other features is left to future work). The
property of independence between two words is inherently global, as it
can be affected by structure arbitrarily far away. Thus we have both
local and global features. The global features are furthermore
distinguished by \textbf{POS level}, explained in detail later. The specific
feature templates are shown below:

\subsubsection{Local Features}
\label{sec-4-1-1}
\begin{enumerate}
\item Left
\label{sec-4-1-1-1}
\begin{itemize}
\item $t_{k-1}$
\item $t_{k-2},t_{k-1}$
\item $t_{k-3},t_{k-2},t_{k-1}$
\end{itemize}

\item Right
\label{sec-4-1-1-2}
\begin{itemize}
\item $t_{k}$
\item $t_{k},t_{k+1}$
\item $t_{k},t_{k+1},t_{k+2}$
\end{itemize}
\end{enumerate}

\subsubsection{Global Features}
\label{sec-4-1-2}

Below, $t^{l}_{i}$ is the $i$ th POS tag in the $l$-level POS tag sequence.

\begin{enumerate}
\item Left
\label{sec-4-1-2-1}
\begin{itemize}
\item $t^l_{i}$ for $1 \le i < k - 1$, $l \in {0,1,2,3}$
\item $t^l_{i},t^l_{i+1}$ for $1 \le i < k - 2$, $l \in {0,1,2,3}$
\item $t^l_{i},t^l_{i+1},t^l_{i+2}$ for $1 \le i < k - 3$, $l \in {0,1,2,3}$
\end{itemize}

\item Right
\label{sec-4-1-2-2}
\begin{itemize}
\item $t^l_{i}$ for $k \le i < n - 1$, $l \in {0,1,2,3}$
\item $t^l_{i},t^l_{i+1}$ for $k \le i < n - 2$, $l \in {0,1,2,3}$
\item $t^l_{i},t^l_{i+1},t^l_{i+2}$ for $k \le i < n - 3$, $l \in {0,1,2,3}$
\end{itemize}
\end{enumerate}

\subsection{POS Level}
\label{sec-4-2}

\begin{table}[tbp]
\centering
\scriptsize

\begin{tabular}{llllllll}
Lvl0 & Lvl1 & Lvl2 & Lvl3 & Lvl0 & Lvl1 & Lvl2 & Lvl3\\
\hline
NN & N & N & N & CD & X & X & \#\\
NNP & N & N & N & -LRB- & X & X & B\\
NNPS & N & N & N & -RRB- & X & X & B\\
NNS & N & N & N & DT & X & X & D\\
PRP & N & N & N & PDT & X & X & D\\
VB & V & V & V & PRP\$ & X & X & D\\
VBD & V & V & V & WP\$ & X & X & D\\
VBG & V & V & V & JJ & X & X & J\\
VBN & V & V & V & JJR & X & X & J\\
VBP & V & V & V & JJS & X & X & J\\
VBZ & V & V & V & -RQ- & X & X & Q\\
, & X & , & , & -LQ- & X & X & Q\\
. & X & . & . & RB & X & X & R\\
: & X & : & : & RBR & X & X & R\\
CC & X & C & C & RBS & X & X & R\\
IN & X & I & I & EX & X & X & X\\
RP & X & I & I & FW & X & X & X\\
TO & X & T & T & LS & X & X & X\\
WDT & X & W & W & MD & X & X & X\\
WP & X & W & W & POS & X & X & X\\
WRB & X & W & W & SYM & X & X & X\\
\# & X & X & \# & UH & X & X & X\\
\$ & X & X & \# &  &  &  & \\
\end{tabular}

\caption{For each POS level, the original tag is replaced with the corresponding value.}
\label{tbl:pos-level}
\end{table}

In previous unpublished work on a similar task, we found that
heuristically transforming the POS tag sequence to create additional
features can be beneficial. We refer to these transformations as \textbf{POS
levels}. In this classifier we implemented three levels, in addition
to the original POS tags as level 0.

We show all levels in Table \ref{tbl:pos-level}. Each level specifies
a value by which each level 0 tag is replaced during the
transformation. The motivation behind each transformation is roughly as follows: level
1 is meant to capture clause nuclei; level 2 is further intended to
show boundaries between clauses; and level 3 expands almost all the
way back to the original tags, but with some distinctions erased,
mostly to reduce the number of features.

\subsection{Which Features Are Useful?}
\label{sec-4-3}

In order to find the best configuration of features for the
classifier, and to evaluate the proposed POS levels, we tested the
classifier using several different combinations. Selected results are
shown in Table \ref{tbl:feature-evaluation}. In the ``Features'' column,
$p$ denotes the local features, and $P_{l}$ denotes the global
features from POS level $l$. 

There are several things worth noting in these results. First, neither
local nor global features are sufficient alone; it appears that local
features promote precision, while global features promote recall.
Second, examining the cases where global features are limited to a
single POS level, it is apparent that each POS level has a different
effect on precision and recall, thus confirming that the classifier is
able to extract different signals from the different POS levels, as
intended. Finally, combining all POS levels together actually reduces
accuracy, likely due to overfitting (although see the discussion of
the kernel classifier below).

\subsection{Results}
\label{sec-4-4}

\begin{table*}[htbp]
%\resizebox{12cm}{!}{

\begin{tabular}{llrrrrrrrrr}
Features & Threshold & Acc & Prec & Rec & F$_{\text{1}}$ & F$_{\text{0.5}}$ & TP & FP & FN & TN\\
\hline
p,P$_{\text{1}}$,P$_{\text{3}}$ & default & 95.44 & 89.05 & 75.74 & 81.86 & 86.03 & 3953 & 486 & 1266 & 32712\\
p,P$_{\text{1}}$,P$_{\text{3}}$ & precision & 94.99 & 91.65 & 69.44 & 79.01 & 86.14 & 3624 & 330 & 1595 & 32868\\
p,P$_{\text{1}}$,P$_{\text{3}}$ & max precision & 92.10 & 95.80 & 43.74 & 60.06 & 77.38 & 2283 & 100 & 2936 & 33098\\
p,P$_{\text{1}}$,P$_{\text{3}}$ & recall & 94.28 & 73.82 & 89.65 & 80.97 & 76.53 & 4679 & 1659 & 540 & 31539\\
\end{tabular}

%}
\caption{Results of classifier using different score thresholds.}
\label{tbl:classifier-results-linear}
\end{table*}

\label{sec:linear-classifier}
For use as input to the parser, we select the $p,P_{1},P_{3}$
feature configuration, and show more detailed results in
Table \ref{tbl:classifier-results-linear}. We used a threshold on the
score output by the classifier to reverse some of the classifier's
decisions in a post-process step. Although it doesn't improve on the
classifier in accuracy, the \texttt{precision} threshold did slightly improve in
F$_{\text{0.5}}$, a measure which favors precision over recall.

\begin{table*}[htbp]
%\resizebox{12cm}{!}{

\begin{tabular}{llrrrrrrrrr}
Features & Threshold & Acc & Prec & Rec & F$_{\text{1}}$ & F$_{\text{0.5}}$ & TP & FP & FN & TN\\
\hline
p,P$_{\text{0}}$,P$_{\text{1}}$,P$_{\text{2}}$,P$_{\text{3}}$ & default & 97.47 & 92.17 & 88.91 & 90.51 & 91.50 & 4640 & 394 & 579 & 32804\\
p,P$_{\text{0}}$,P$_{\text{1}}$,P$_{\text{2}}$,P$_{\text{3}}$ & precision & 97.27 & 92.95 & 86.43 & 89.58 & 91.57 & 4511 & 342 & 708 & 32856\\
p,P$_{\text{0}}$,P$_{\text{1}}$,P$_{\text{2}}$,P$_{\text{3}}$ & max precision & 96.57 & 94.22 & 79.63 & 86.31 & 90.89 & 4156 & 255 & 1063 & 32943\\
p,P$_{\text{0}}$,P$_{\text{1}}$,P$_{\text{2}}$,P$_{\text{3}}$ & recall & 97.15 & 88.16 & 91.32 & 89.71 & 88.78 & 4766 & 640 & 453 & 32558\\
\end{tabular}

%}
\caption{Results of polynomial classifier using different score thresholds.}
\label{tbl:classifier-results-poly}
\end{table*}

\subsection{Polynomial Kernel}
\label{sec-4-5}

\label{sec:poly-classifier} For comparison with the linear classifier,
we trained another classifier using a polynomial kernel (with
degree 3) with all the features. The results are shown in Table
\ref{tbl:classifier-results-poly}. The polynomial kernel improves over
the linear classifier in accuracy by 2\%, in precision by 3 points, and
in recall by just over 13 points. This suggests that there is a large
potential for improving the linear classifier by adding conjunctive
features. Alternatively, there are methods for effectively linearizing
a kernel-based classifier \cite{Kudo2003,Isozaki2002}.
Currently, the polynomial classifier takes over 2 hours to run on
section 22 (training the model took almost 4 days).



\section{Parsing With Independence Constraints}
\label{sec-5}
\label{sec:parser}

In order to demonstrate use of the independent constraints in a
parser, we modified the CKY parser included in the Stanford Parser
distribution to accept independent span boundaries as constraints and
to use the modified CKY algorithm described above. Our modifications
are:

\begin{itemize}
\item after reading in the grammar, index the synthetic binary rules
\item read in the file containing the boundaries output by the classifier
from the previous section
\item for each CKY cell, if the cell spans a boundary then loop over just
the synthetic binary rules
\item if at the end of the CKY loop a parse was not successful, then loop
again over just the cells which span a boundary and process all of
the binary rules
\item output the total number of times entering the inner loop as well as the
number of times the parser failed
\end{itemize}

\subsection{Experimental Setup}
\label{sec-5-1}

We use the modified Stanford Parser described above, with a grammar
extracted from the WSJ sections 02-21, and evaluate its performance on
section 22 using output from the clasifier as constraints. We vary the
threshold on the probability output by the classifier, and further
experiment with restricting the constraints to sentences above a
certain length. Finally, to compare with previous results we run the
classifier and parser on section 23 in a single configuration.

All experiments were run on a DELL Precision 690, with 8 cores and 32G
of RAM. Unless otherwise noted multiple processes were run in
parallel, and times reported were not averaged over multiple runs.
Since we saw significant variation of up to 10\%, the times should be
taken with a grain of salt. The computation done in the CKY algorithm
is measured in the number of binary edges visited in the inner loop. A
binary edge is a tuple of a span (begin \& end), a binary rule $A →
BC$, and a split point (the position where $B$ and $C$ meet).

\subsection{Results}
\label{sec-5-2}

\begin{table*}[tbp]
%\resizebox{12cm}{!}{

\begin{tabular}{rlrllr}
SentLen & Constraints & Time (s) & \# Edges & F$_{\text{1}}$ & Parse Failures\\
\hline
0 & default & 1283 & 1.08\texttimes{}10$^{\text{10}}$ (62\%) & 83.71 (-2.14) & 15\\
0 & precision & 1143 & 1.13\texttimes{}10$^{\text{10}}$ (65\%) & 84.05 (-1.80) & 7\\
0 & max precision & 1384 & 1.42\texttimes{}10$^{\text{10}}$ (81\%) & 85.55 (-0.30) & 2\\
0 & recall & 1024 & 7.80\texttimes{}10$^{\text{09}}$ (45\%) & 78.74 (-7.11) & 136\\
20 & default & 1126 & 1.12\texttimes{}10$^{\text{10}}$ (64\%) & 84.17 (-1.68) & 9\\
20 & precision & 1313 & 1.16\texttimes{}10$^{\text{10}}$ (66\%) & 84.43 (-1.42) & 4\\
20 & max precision & 1338 & 1.44\texttimes{}10$^{\text{10}}$ (82\%) & 85.59 (-0.26) & 2\\
20 & recall & 1121 & 8.24\texttimes{}10$^{\text{09}}$ (47\%) & 80.38 (-5.47) & 103\\
30 & default & 1312 & 1.28\texttimes{}10$^{\text{10}}$ (73\%) & 84.82 (-1.03) & 3\\
30 & precision & 1279 & 1.31\texttimes{}10$^{\text{10}}$ (75\%) & 85.01 (-0.84) & 1\\
30 & max precision & 1485 & 1.53\texttimes{}10$^{\text{10}}$ (87\%) & 85.63 (-0.22) & 1\\
30 & recall & 1140 & 1.02\texttimes{}10$^{\text{10}}$ (58\%) & 82.79 (-3.06) & 57\\
40 & default & 1476 & 1.51\texttimes{}10$^{\text{10}}$ (86\%) & 85.56 (-0.29) & 1\\
40 & precision & 1390 & 1.52\texttimes{}10$^{\text{10}}$ (87\%) & 85.59 (-0.26) & 0\\
40 & max precision & 1513 & 1.65\texttimes{}10$^{\text{10}}$ (94\%) & 85.75 (-0.10) & 0\\
40 & recall & 1403 & 1.33\texttimes{}10$^{\text{10}}$ (76\%) & 84.65 (-1.20) & 14\\
\hline
∞ & baseline & 1558 & 1.75\texttimes{}10$^{\text{10}}$ (100\%) & 85.85 & 0\\
\end{tabular}

%}
\caption{Independence constraints reduce the work done by the CKY algorithm, trading off accuracy.}
\label{tbl:parse-results-linear}
\end{table*}

The results of running the parser on section 22 using the linear
classifier from Section \ref{sec:linear-classifier} are shown in
Table \ref{tbl:parse-results-linear}. The table shows the total time
taken, the total times entering the inner loop, the F$_{\text{1}}$ and difference
from the baseline, and the number of times the parse failed using the
constraints. The baseline consisted of the same parser with the
sentence length threshold set to 1000. The time includes the time
spent reading in the constraints but not the time taken by the
classifier.

The parser with the independence constraints saves 35-38\%
of the computation inside the CKY loop over the baseline,
corresponding to about 20\% reduction in total time, at the cost of a
2-point drop in F-score. After increasing recall by making negative
instances for which the classifier assigned a low probability positive,
the parser reduced the work done inside the loop to less than half the
baseline, but accuracy also plummeted by 7 points.

\subsection{Polynomial Kernel}
\label{sec:results-poly}

A difference of 2 F$_{\text{1}}$ score is not small, but on the other hand it is
about by how much the unlexicalized Stanford Parser trails the Collins
parser, for example. However, as shown above in Section
\ref{sec:poly-classifier}, there is room to improve the linear
classifier through conjunctive features. As an indication of an upper
bound of the acheivable performance, we tried using the output of the
kernel classifier in the parser as above, while acknowledging that at
present the time needed to produce the classifier output dwarfs the
time needed to actually parse the test data.

The results of running the parser on section 22 with the polynomial
classifier output are shown in Table \ref{tbl:parse-results-poly}.
With the more accurate classifier, the parser is able to reduce the
necessary computation even further, by about 45\%, while losing less
accuracy. With a high-precision threshold, the computation of the CKY
algorithm is reduced to less than 60\% of the baseline, while losing
less than half a point F$_{\text{1}}$ score.

\begin{table*}[tbp]
%\resizebox{12cm}{!}{

\begin{tabular}{rlrllr}
SentLen & Constraints & Time (s) & \# Edges & F$_{\text{1}}$ & Parse Failures\\
\hline
0 & default & 1106 & 9.74\texttimes{}10$^{\text{09}}$ (56\%) & 84.85 (-1.00) & 6\\
0 & precision & 1118 & 9.84\texttimes{}10$^{\text{09}}$ (56\%) & 85.12 (-0.73) & 4\\
0 & max precision & 1137 & 1.02\texttimes{}10$^{\text{10}}$ (58\%) & 85.42 (-0.43) & 2\\
0 & recall & 1050 & 9.25\texttimes{}10$^{\text{09}}$ (53\%) & 84.05 (-1.80) & 33\\
20 & default & 1070 & 1.02\texttimes{}10$^{\text{10}}$ (58\%) & 85.08 (-0.77) & 5\\
20 & precision & 1172 & 1.03\texttimes{}10$^{\text{10}}$ (59\%) & 85.25 (-0.60) & 3\\
20 & max precision & 1092 & 1.06\texttimes{}10$^{\text{10}}$ (61\%) & 85.41 (-0.44) & 2\\
20 & recall & 1088 & 9.68\texttimes{}10$^{\text{09}}$ (55\%) & 84.75 (-1.10) & 7\\
30 & default & 1222 & 1.20\texttimes{}10$^{\text{10}}$ (69\%) & 85.57 (-0.28) & 1\\
30 & precision & 1267 & 1.20\texttimes{}10$^{\text{10}}$ (69\%) & 85.62 (-0.23) & 1\\
30 & max precision & 1238 & 1.23\texttimes{}10$^{\text{10}}$ (70\%) & 85.65 (-0.20) & 1\\
30 & recall & 1238 & 1.16\texttimes{}10$^{\text{10}}$ (66\%) & 85.44 (-0.41) & 2\\
40 & default & 1465 & 1.49\texttimes{}10$^{\text{10}}$ (85\%) & 85.72 (-0.13) & 0\\
40 & precision & 1353 & 1.49\texttimes{}10$^{\text{10}}$ (85\%) & 85.75 (-0.10) & 0\\
40 & max precision & 1570 & 1.50\texttimes{}10$^{\text{10}}$ (86\%) & 85.78 (-0.07) & 0\\
40 & recall & 1489 & 1.47\texttimes{}10$^{\text{10}}$ (84\%) & 85.69 (-0.16) & 1\\
\hline
∞ & baseline & 1470 & 1.75\texttimes{}10$^{\text{10}}$ (100\%) & 85.85 & 0\\
\end{tabular}

%}
\caption{The classifier using the polynomial kernel is much more accurate, leading to smaller loss in accuracy of the parser.}
\label{tbl:parse-results-poly}
\end{table*}

\subsection{WSJ Section 23}
\label{sec-5-4}

To compare with previous work on parsing using the Penn Treebank, we
show the time and accuracy for parsing section 23, using both linear
and kernel classifier output, along with the baseline parser, below.
The times reported are the average of three runs each. Because there
was significant variation in parse time when multiple processes were
run in parallel, for these results only one process was run at a time.
The results parallel those shown on section 22.

\begin{center}
\begin{tabular}{rc@{}lrr}
Parser & Time (s) &  & F$_{\text{1}}$ & \\
\hline
baseline & 1538 &  & 85.54 & 0.00\\
linear & 1106 & \texttimes{}1.39 & 83.55 & -1.99\\
poly & 1040 & \texttimes{}1.48 & 84.57 & -0.97\\
\end{tabular}
\end{center}

\section{Related Work}
\label{sec-6}

There are several strains of research related to adding constraints to
the CKY chart. \cite{Roark2012} describes an approach using
finite-state taggers to decide whether each word in a sentence begins
or ends a multiword constituent and has a unary span or not. They show
that their tagger is able to achieve very high precision, reducing
parse time without negatively affecting accuracy.

\cite{Bodenstab2011} proposes a classifier which directly decides for
each cell in the chart how many constituents should be created. Their
parser uses beam search with a FOM and a beam for each chart cell.

Like these approaches, our method uses a classifier to avoid doing
work in certain chart cells. While not completely orthogonal, we
believe our independence constraints are complementary. A single
decision by our classifier closes a large swath of cells based on the
global structure, while their methods make local decision using local
information.

\cite{Yarmohammadi2014} proposes a concept of 'hedge' parsing, where
only spans below a certain length are allowed, and show how this
reduces the computation done by CKY. Their approach for segmenting a
sentence and parsing within each segment separately is similar to the
approach presented in this paper; however, their system does not create
spans of length larger than the threshold and thus doesn't follow the
original treebank annotation, while our approach is able to return the
original gold parse tree provided that the classifier does not output
a false positive.

\section{Conclusions}
\label{sec-7}

We have proposed a property of \textbf{independence} between words in a
sentence, and shown how to use this property to create top-down
constraints which can be used to reduce the computation done by the
CKY algorithm. We demonstrated two classifiers for identifying
boundaries between independent words given a sentence with only
surface features, a linear classifier which is fast but less accurate,
and a classifier with a polynomial kernel which is much more accurate
but very slow. We then showed that a commonly-used CFG parser can be
made faster by using the output of these classifiers to create
top-down constraints at the cost of some accuracy, which can be
traded-off by varying the confidence threshold of the classifier
results.

Although the loss of accuracy when using the linear classifier is
currently uncomfortably large, the performance of the kernel
classifier indicates that there is room for improvement, either by
manually adding conjunctive features to the linear classifier or using
a method to automatically linearize the model. Features based on words
as well as POS tags may also be beneficial. However, the current
approach has several weaknesses which should be addressed by future
research.

First, the top-down nature of the independence constraints does not
make a natural fit with the bottom-up CKY algorithm. In particular,
the binary nature of the rules in the grammar combined with the
bottom-up search means that the parser still ends up doing some
computation to create spans which violate the constraints, even though
it is prevented from completing such a span.

Second, the pipelined nature of the classifier means that it only has
access to POS tags and in particular is not able to make use of
information generated as the parser processes lower-level spans.
Tighter integration of the classifier into the parser may be
beneficial to both.

Third, the current classifier combines instances from different
syntactic structures into a single model. It is possible that training
multiple models on different types of sentences would result in a
better classifier.

\bibliographystyle{acl}
\bibliography{references}
% Emacs 24.4.1 (Org mode 8.2.10)
\end{document}